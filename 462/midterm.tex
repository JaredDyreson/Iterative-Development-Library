\documentclass{article}

\usepackage{mathrsfs,amsmath}
\usepackage{xcolor}
\usepackage{titlesec}

\usepackage{graphicx}

\graphicspath{ {./assets/} }

\usepackage[margin=1.4in]{geometry}

\title{Lab \#4 | CS 462} 
\author{Jared Dyreson\\ 
        California State University, Fullerton}

\DeclareRobustCommand{\bowtie}{%
  \mathrel\triangleright\joinrel\mathrel\triangleleft}


\usepackage [english]{babel}
\usepackage [autostyle, english = american]{csquotes}
\MakeOuterQuote{"}

\titlespacing*{\section}
{0pt}{5.5ex plus 1ex minus .2ex}{4.3ex plus .2ex}
\titlespacing*{\subsection}
{0pt}{5.5ex plus 1ex minus .2ex}{4.3ex plus .2ex}

\usepackage{hyperref}
\hypersetup{
    colorlinks,
    citecolor=black,
    filecolor=black,
    linkcolor=black,
    urlcolor=black
}

\begin{document}

\maketitle
\tableofcontents
\newpage

\section{Inception Phase}

\subsection{Use Case}

\textbf{Definition} | text stories, widely used to discover and record requirements. These are not diagrams.


\newpage

\subsection{Quiz with Answers}

\begin{enumerate}

\item A fully dressed use case contains the following:
\begin{itemize}
\item Use Case Name | starts with a verb
\item Scope | system under design
\item Level | "user-goal" or "sub-function"
\item Primary Actor | Calls on the system to deliver its services
\item Stakeholders and Interests | Who cares about this use case, and what do they want?
\item Preconditions | What must be true on start, \emph{and} worth telling the reader?
\item Success Guarantee | What must be true on successful completion, \emph{and} worth telling the reader
\item Main Success Scenario | A typical, unconditional happy path scenario of success
\item Extensions | Alternate scenarios of success or failure
\item Special requirements | Related, non-functional requirements
\end{itemize}

\item Each phase and iteration has some \underline{\textbf{placeholder}} focus, and concludes with a well-defined milestone.

\item Guidelines for writing use cases:
\begin{itemize}
\item Use the Elementary Business Process (EBP) Test | how much money will this make me?
\item Use the "Boss Test" | "what have you been doing all day?" should not be met with "not a whole lot"
\item Use the Size Test | is this artifact too small or too big? Does it convey what I need it to?
\end{itemize}

\item Fully dressed use case terminology:
\begin{itemize}
\item Scope | bounds of the system under design
\item Success Guarantee | states in a system that must be true for the system to complete successfully in that case
\item Extensions | Alternate success/failure scenarios (authentication, sufficient funds)
\item Primary Actor | Calls on the system to deliver its services (customer withdrawing money from account)
\item Stakeholders and Interests | Who cares about this use case, and what do they want? 
\item Preconditions | What must be true on start, and worth telling the reader (Sufficient funds to complete the transaction are required)
\item Main Success Scenario | A typical, unconditional happy path scenario of success (Customer can pay their bills)
\end{itemize}

\newpage

\item Fully dress use case terminology (continued):
\begin{itemize}
\item Domain Rules | dictates how a domain or business will operate
\item Use case(s) | collection of related success and failure scenarios that describe an actor using a system to support a goal.
\item Vision | an executive summary that briefly describes the project as a context for the major players to establish a common understanding of the project
\item Actor | something with a behavior mechanism, such as a computer system, person or organization
\item Scenario | a specific sequence of actions and interactions between actors and the system (\textbf{use case instance})
\item Use Case Instance | scenario
\item Use Case Model | the set of all written use cases
\end{itemize}

\item The most important thing you can learn in object-oriented analysis and design (OOAD) is \underline{\textbf{objects are responsible for their own actions | responsibility assignment}}\footnote{See Pg. 6}. More information can be found \href{https://www.umsl.edu/~sauterv/analysis/488_f01_papers/quillin.htm}{\underline{here}}. 

\item Use cases can be written in different formats and levels of formality. Match the use case format with its description:

\begin{itemize}
\item Fully dressed | all steps and variations are written in detail, and there are supporting sections
\item Brief | terse, one-paragraph summary, usually of the main success scenario
\item Casual | Informal paragraph(s) format that covers various scenarios
\end{itemize}

\item Match the UML Perspective with its description

\begin{itemize}
\item More information can be found \href{http://etutorials.org/Programming/UML/Chapter+4.+Class+Diagrams+The+Essentials/Perspectives/}{\underline{here}}
\item Specification Perspective | Diagrams describe software abstractions or components with specifications and interfaces, but no commitment to a particular implementation
\item Conceptual Perspective | Diagrams are interpreted as describing things in a situation of the real world or domain interest
\item Implementation Perspective | Diagrams that describe software implementations in a particular technology
\end{itemize}

\item Use cases are defined to satisfy goals of the primary actors. The basic procedure to find meaningful use cases for these actors are listed below:

\begin{itemize}
\item Choose the system boundary
\item Identify the primary actors
\item Identify the goals for each primary actor
\item Define use cases that satisfy user goals; name them according to their goal
\end{itemize}

\item The phases of the Unified Process (UP) are equivalent to the phases of the traditional waterfall process model: \textbf{FALSE}.
\item Should you analyze the Inception Phase artifacts thoroughly during Inception? |  \textbf{NO}, should be brief and rough.

\item Use cases can be characterized by which of the following:
\begin{itemize}
\item Are text documents
\item Are widely used to discover and record requirements
\item Describe the system's operational behavior as used by it primary actors
\end{itemize}

\item Which should be written first, use cases or the vision: \textbf{VISION}
\item Object-Oriented Analysis can be characterized by\footnote{See Pg 7}:

\begin{itemize}
\item An investigation of the problem, rather than how a solution is defined
\item Emphasizes a conceptual solution that fulfills the requirements
\item Do the right thing (measure twice)
\end{itemize}

\item There are three kinds of external actors. Match the word or phrase with its description
\begin{itemize}
\item Offstage Actor | has an interest in the behavior of the use case, but not directly involved
\item Supporting Actor | provides a service to the system under discussion
\item Primary Actor | has use goals fulfilled through using the services of the system under discussion
\end{itemize}

\item The Supplementary Specification captures non-functional requirements, information, and constraints not easily captured in the use cases. Which of the elements below are likely to be included in the Supplementary Specification?\footnote{See Pg 107}
\begin{itemize}
\item Installation, administration, and help documentation
\item Reports and report formats
\item Quality attributes | Usability, reliability, performance, and supportability
\item Development constraints
\item Hardware and software constraints 
\end{itemize}
\item What are good rules of thumb for answering the following: \emph{What is the useful level to express use cases for application requirement analysis?}:
\begin{itemize}
\item Boss Test | "What have you been doing all day?!"
\item EBP Test | Events that adds measurable business value (more \$)
\item The Size Test | Did you try to explain too much in a small amount of text? Is this text too pedantic for such a small step?
\end{itemize}

\item Object Oriented Design can be characterized by which of the following?
\begin{itemize}
\item Do the thing right (cut once)
\item Defining software objects and how they collaborate to fulfill the requirements
\end{itemize}

\item What artifacts are typically started in the Inception Phase?\footnote{See Pg. 50}:
\begin{itemize}
\item Vision and Business Case
\item Use-Case Model
\item Supplementary Specification
\item Glossary
\item Risk List/Registry
\item Prototypes and PoCs (Not in list)
\item Iteration Plan (Not in list)
\item Phase Plan \& Software Development Plan (Not in list)
\item Development Case (Not in list)
\end{itemize}

\item What are the Unified Process phases?
\begin{itemize}
\item Inception Phase
\item Elaboration Phase
\item Construction Phase
\item Transition Phase
\end{itemize}
\end{enumerate}

\end{document}
