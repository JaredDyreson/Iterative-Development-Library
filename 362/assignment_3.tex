\documentclass{article}

\usepackage{mathrsfs,amsmath}
\usepackage{xcolor}
\usepackage{titlesec}
\usepackage{minted}
\usepackage{mwe}

\usepackage[margin=1.4in]{geometry}

\title{Assignment \#3 | CS 362 Software Engineering}
\author{Jared Dyreson \\
        jareddyreson@csu.fullerton.edu \\
        California State University, Fullerton}
\date

\DeclareRobustCommand{\bowtie}{%
  \mathrel\triangleright\joinrel\mathrel\triangleleft}


\usepackage [english]{babel}
\usepackage [autostyle, english = american]{csquotes}
\MakeOuterQuote{"}

\titlespacing*{\section}
{0pt}{5.5ex plus 1ex minus .2ex}{4.3ex plus .2ex}
\titlespacing*{\subsection}
{0pt}{5.5ex plus 1ex minus .2ex}{4.3ex plus .2ex}

\usepackage{hyperref}
\hypersetup{
    colorlinks,
    citecolor=black,
    filecolor=black,
    linkcolor=black,
    urlcolor=black
}

\begin{document}

\maketitle
\tableofcontents

\newpage

\section{Chapter 8 | Software Quality Assurance}
This process covers a multitude of points such as:

\begin{itemize}
\item Quality Management Process
\item Effective software engineering technology (methods and tools)
\item Formal technical reviews that are applied throughout the software process
\item Multitiered testing strategy
\item Control of software documentation and the changes made to it
\item A procedure to ensure the compliance with software development standards (when applicable)
\item Measurement and reporting mechanisms
\end{itemize}

All of these actions attempt to ensure the quality and effectiveness of the software produced.
There are also different axes in which quality can be measured such as design and conformance.
These procedures also can be fully automated, manual or a hybrid of the two.
During the process, data should be taken to help further improve future endeavors.
Removing errors that occur in this process is less costly and locks down the possibility of catastrophic consequences in production.

\section{Chapter 9 | Software Configuration Management}

\emph{Software configuration management} (SCM) activities include:

\begin{enumerate}
\item Identify change
\item Control change
\item Ensure that change is being properly implemented
\item Report changes to others who may have an interest 
\end{enumerate}

This is the basis for version control systems such as Git.
Git was created by Linus Torvalds during the early development of the Linux Kernel.
It is important for a software engineering team to work in sync with one another and being able to revert/apply changes to a codebase is paramount to this.
While Git is a piece of software that we use (considered automated per Chapter 8), the most effective part comes when technical reviews are conducted.
Given the proper information, members in a group can allow or disallow these changes.

\end{document}
